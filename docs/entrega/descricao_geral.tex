% Created 2020-10-20 Tue 03:52
% Intended LaTeX compiler: pdflatex
\documentclass[11pt, a4paper]{article}
\usepackage[utf8]{inputenc}
\usepackage[T1]{fontenc}
\usepackage{graphicx}
\usepackage{grffile}
\usepackage{longtable}
\usepackage{wrapfig}
\usepackage{rotating}
\usepackage[normalem]{ulem}
\usepackage{amsmath}
\usepackage{textcomp}
\usepackage{amssymb}
\usepackage{capt-of}
\usepackage{hyperref}
\usepackage[left=3cm, top=3cm, right=2cm, bottom=2cm]{geometry}
\usepackage[brazilian, ]{babel}
\usepackage{indentfirst}
\author{Leon Ferreira Bellini}
\date{\today}
\title{}
\hypersetup{
 pdfauthor={Leon Ferreira Bellini},
 pdftitle={},
 pdfkeywords={},
 pdfsubject={},
 pdfcreator={Emacs 27.1 (Org mode 9.4)}, 
 pdflang={Breton}}
\begin{document}


\begin{titlepage}
\begin{center}
Centro Universitário FEI
\end{center}
\vspace*{\fill}
\begin{center}
  \huge{Avai\textbf{Lab}le: Informativo quanto as vagas disponíveis nos laboratórios CGI}
\end{center}
\vspace*{\fill}
  \Large{Leon Ferreira Bellini} \\
  \small{22218002-8} \\\\
  \Large{Guilherme Ormond Sampaio} \\
  \small{22218007-7}
\end{titlepage}

\tableofcontents

\clearpage

\section{Introdução}
\label{sec:org42fe0eb}
Diante de uma época considerada ``normal'' quando se refere as atividades do
\textbf{Centro Universitário FEI}, muitos dos alunos os quais dependem dos vários
laboratórios disponíveis para livre uso, certamente notaram que estes
se encontravam 
lotados, principalmente durante os períodos de provas \textbf{P1}
(para os cursos que ainda as possuem) e \textbf{P2}.

Para resolver tal problema, imaginando um indivíduo vinculado à empresa,
o qual possui o trabalho de resolver problemas genéricos, este pode
pensar nas seguintes possibilidades:

\begin{enumerate}
\item Adquirir novas unidades de computadores, mesas e cadeiras
(cabos e instalação inclusos no cenário)
\item Construir um novo laboratório (cenário 1 implícito, mais caro)
\item Desenvolver um \textbf{sistema} para, ao menos, refrear a superlotação dos laboratórios
a partir da realização de uma gestão propriamente dita, levando em conta
o comportamento dos \textbf{usuários finais} (alunos).
\item Limitar a entrada de usuários, impondo condições severas em relação ao
uso das máquinas e lugares disponíveis 
(não mais que \textbf{n} pessoas presentes ou uma máquina a cada duas pessoas são possíveis restrições)
\end{enumerate}

Deve-se lembrar, entretanto, que existem finitas soluções possíveis as quais
englobam tanto curto, quanto longo prazo. Porém, o grupo, defronte ao
potencial da opção quatro (4), decidiu implementar um sistema de software
capaz de realizar tal ação. 

\section{Discussão}
\label{sec:org4c57f53}
Alinhado aos princípios da empresa como entrega de conhecimento de forma
ágil, digital e dinâmica está presente um dos métodos 
de desenvolvimento de \emph{software} ágil, o chamado \textbf{XP} ou \textbf{Extreme Programming}
o qual se baseia num ciclo simples onde relatos (ou histórias) dos usuários,
têm extrema relevância, estas
entrando em foco para "moldar" o produto final (\ref{org376bcb0}). Será mesclado
a este processo os métodos já utilizados pelo grupo em experiencias anteriores,
estes serão explicitados em seções posteriores.

\subsection{A escolha da Programação Extrema}
\label{sec:org7a53bc8}
Por ser um método ágil, contendo ``apenas'' quatro passos para a implementação
propriamente dita de um componente e, tendo como parte fundamental da fase de
\textbf{codificação} a programação em duplas, o \textbf{XP} foi escolhido por cobrir a
o fato de que a equipe não conta com mais do que, exatamente, dois integrantes.
Qualquer
outro método não foi enxergado como apropriado com o estilo de trabalho da
equipe. Além disso, a proximidade com o usuário ao incluir suas histórias como
elemento essencial para a produção dos \textbf{cartões CRC} gera a oportunidade de
desenvolver o melhor produto para tais indivíduos. Pode-se realizar a comparação
com um acionista, por exemplo. Um usuário o qual tenha exigências
quanto a um futuro \emph{software} cria sua história (investe), esperando que
suas experiências tenham um retorno direto (o \emph{software}) e neste caso,
devido à agilidade do \textbf{XP}, a entrega é rápida.

\subsection{O \emph{workflow} do grupo}
\label{sec:orgbb00d6a}
Como dito anteriormente, a equipe ``adaptou'' o método \textbf{XP} para melhor trabalhar
durante os quatro meses de desenvolvimento. A forma pela qual ocorreu estas
mudanças serão informadas nas próximas seções. De forma de se conter no ciclo
de desenvolvimento do \textbf{XP}, a equipe realiza encontros a cada três dias na
plataforma \textbf{VOIP} \emph{Discord} (\ref{org2d92571}) para cumprir com cada responsabilidade gerada
pelas primeiras \textbf{etapas}.

\subsubsection{Planejamento}
\label{sec:org98c08c4}
Pode-se traçar uma relação com a fase de \textbf{coleta de requisitos}, porém esta
de forma ``mini'', cada usuário entrevistado (estes sendo, em sua totalidade,
estudantes da FEI), relata suas experiências quando interagindo com o universo
em que o \emph{software} terá influência futuramente. É então, realizada para eles,
a pergunta, "O quão relevante é para você que esta característica seja implementada?".
Uma resposta, por exemplo, sendo "De extrema relevância", possivelmente
garantiria a uma dessas ``features'' prioridade também extrema. Similarmente,
prioridades de pesos diferentes são dadas a relatos de mesmo peso. Logo,
esta fase determina o que será trabalhado pelos próximos dias. 

Devido ao tempo limitado que a equipe possui, ainda ocorre um processo de
filtragem durante as reuniões, nas quais a dupla escolhe dentre as consideradas
"mais relevantes" as que são essenciais para, pelo menos, apresentação do projeto.

Além do mais, não são requisitadas dos usuários divisões de histórias em ``sub-histórias''
e o peso dado para cada história determinada nos cartões é em \textbf{dias}.  

\subsubsection{Projeto}
\label{sec:orgd29618a}
Essenciais para a determinação do que será programado pela equipe na próxima
etapa, os cartões \textbf{CRC} (Classe-Responsabilidade-Colaborador) são concebidos 
ao se basear nos componentes definidos à partir das histórias dos usuários
da fase anterior. A cada reunião de projeto, é feito um cartão \textbf{CRC} em
um arquivo \textbf{org}, linguagem de \emph{markdown} a qual facilita a formatação de
documentos destinados à gestão de tempo por \emph{scheduling}, listas \emph{TODO} e
planilhas. O arquivo, então, fica à disposição para futuras consultas.
O \textbf{org-mode} é parte do editor de texto GNU/Emacs (\ref{orge0df9e4}),
o qual é utilizado pela equipe
em todas as fases. A partir do momento que este cartão se apresenta terminado,
é adicionado um novo componente no \textbf{diagrama de componentes} o qual será
apresentado em relatório posterior, bem como uma nova classe relacionada
ao novo componente. É correto afirmar que esta fase define como será implementado
o componente em foco de forma técnica, uma vez que, acompanhado ao cartão CRC,
está a elaboração da classe em si no sistema orientado à objetos que o projeto
incluí.

A dupla não cria protótipos. 

\subsubsection{Codificação}
\label{sec:orgbc16847}
Uma vez que a dupla de programadores também forma a equipe por completo,
todo o processo de criação de testes unitários (o conjunto de \emph{stubs} e \emph{drivers})
e o ato de codificação em si ocorre ``de uma vez'', a implementação, na maioria das
vezes, é concluída em uma e, raramente, duas reuniões de codificação. O arquivo \textbf{org} de
cronograma é, então, atualizado ao fim dos testes.

\subsubsection{Testes}
\label{sec:org1aa3510}
Possivelmente a fase mais importante para o desenvolvimento de um sistema
robusto, pode-se dizer que, para o grupo, a fase de testes é \textbf{constante}, dado
que a fase de codificação já incluí o teste unitário como essencial. Nesta
fase, a equipe também aplica o que foi determinado no documento de
\textbf{Especificação de teste}. A fase de testes assegura que o grupo possa ``zerar''
o ciclo e começar uma nova fase de planejamento, já que, isoladamente, o
componente produzido não apresenta mais erros ou \emph{bugs}. O grupo escolheu
por apenas realizar o teste \emph{top-down} ao final da programação de todos
os componentes de um \emph{package}, este definido no \textbf{diagrama de componentes}.


\clearpage   
\section{Bibliografia}
\label{sec:org7296298}
\begin{enumerate}
\item \label{org376bcb0}
\label{sec:orgb6bc9f4}
PRESSMAN, Roger S. Engenharia de software: Uma abordagem profissional. 7. ed. Porto Alegre: AMGH Editora, 2011.
\item \label{orge0df9e4}
\label{sec:org4375ff5}
GNU Emacs. Disponível em:< \url{https://www.gnu.org/software/emacs/} >. Acesso em 20 outubro 2020.
\item \label{org2d92571}
\label{sec:org1d2e3c9}
DISCORD. Disponível em:< \url{https://discord.com/} >. Acesso em 20 outubro 2020.
\end{enumerate}
\end{document}